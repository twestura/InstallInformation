\documentclass{article}

\author{Travis Westura}
\title{New Computer Software Setup}
\date{\today}

\usepackage{fullpage}
\usepackage{hyperref}
\usepackage{cleveref}

% This macros broke when a url contained a #.
\newcommand{\web}[2]{
	\begin{center}
		\url{#1}{#2}
	\end{center}
}

\begin{document}

\maketitle

Whenever I get a new computer I spend time installing a variety of software.
This guide is intended to serve as a reminder about what I install and how to install it.
In writing this guide I hope to save myself much time going through StackOverflow posts.

\section{Web Browsers}

First we need to find something better than Internet Explorer.
Firefox is downloaded from here: 
	\web{https://www.mozilla.org/en-US/firefox/new/}{.}
Google Chrome is downloaded from here:
	\web{https://www.google.com/chrome/browser/}.

\section{Git}
Git can be downloaded from here:
	\web{http://git-scm.com/downloads}.
Run the installer.
After installing open a git bash shell.
Run the commands git config --global user.name "Travis" and git config --global user.email "twestura@gmail.com".

Optionally we could install GitHub for Windows and do something to save my password.

\section{Sublime Text 2}
I use Sublime Text 2 without a license, as doing so is free and the annoying pop-up message is infrequent.
It can be downloaded here:
	\web{http://www.sublimetext.com/2}.
Just download the Windows 64 bit version.
Simply run the installer after downloading.

\section{Emacs}
Ahh yes: Emacs.
Let's get it running on Windows.
I downloaded Emacs by using the link on the instructions for setting up \LaTeX on Windows here:
	 \web{http://www.latexbuch.de/install-latex-windows-7/\#x1-110003.1}.
\footnote{
	This version is probably old.
	There is likely a new version simply on the emacs website.
	There is no installer required, so I should be able to unzip the zip folder and run it.
	I am not sure what a precompiled version, but I assume that I would need to compile it.
	Anyway I should be able to use a version that is not precompiled.

	Actually, it seems as though the link is updated to the latest version.
	However, it would not hurt to check whenever I want to download something.
}
I unzip everythin on my desktop and create a folder called emacs in my Program Files.
Copy everything from the unzipped file into this folder.
There is an installer addpm.exe in bin that can be run, although I am not sure what it does.

I may then need to set the HOME environment varaible to be C:\textbackslash{}Users\textbackslash{}Travis,
	as described here:
	\web{https://www.gnu.org/software/emacs/manual/html_node/emacs/Windows-HOME.html}.
I may also want to add emacs to my path variable.
To run emacs in the command shell, the command is emacs -nw, with the nw meaning nonwindowed.

I will also need to create a .emacs file.
I cannot create file names that begin with periods in a file explorer.
But I can simply create the file using emacs.
To add line numbers by default, add the line (global-linum-mode 1) to the .emacs file.

For instructions on how to set up \LaTeX with emacs, see \cref{subsubsec:LaTeX_Emacs}.

\section{\LaTeX}

\subsection{Installation}

I use the TexLive distribution that can be obtained here:
	\web{https://www.tug.org/texlive/acquire-netinstall.html}.
Use the install-tl-windows.exe installer.
The installation can take quite some time.

I use TexLive to be consistant across multiple platforms, as MikTex only supports Windows.

There may be something else that needs to be installed called getnonfreefonts.
The link is \web{https://www.tug.org/fonts/getnonfreefonts/}.
Download the file and navigate to the directoy containing it in a command line.
Then run texlua install-getnonfreefonts getnonfreefonts-sys-a.

\subsection{Text Editors}

I use several text editors with \LaTeX.
In the following I describe how to get \LaTeX to work with these editors.
Other installation and setup instructions for these editors can be found in the respective sections for the editors.

\subsubsection{Sublime}

\subsubsection{Emacs}
\label{subsubsec:LaTeX_Emacs}

To use \LaTeX, we want to set up AUC\TeX.
We can download it here:
	\web{http://www.gnu.org/software/auctex/download-for-windows.html}.
Unzip the folder and make sure the contents are contained in a folder called auctex.
Then make sure this folder is contained inside the emacs folder.
A simple test pdf document can then be compiled using C-c C-c RET.
These instructions by themselves are not enough.
Moving the Auc\TeX folder away still allows me to complile the same way.
Thus Auc\TeX is likely not installed correctly.
Okay, so when I download Auc\TeX, I need to unzip the files directly into the emacs folder.
There will not be an emacs folder containing an auctex folder.
Rather the contents directly overwrite whatever is in the emacs folder.

Several sites said to install Ghostscript, so I downloaded it from here:
	\web{http://www.ghostscript.com/download/gsdnld.html}.
I set up a GhostScript printer with the instructions from here:
	\web{http://www.latexbuch.de/install-latex-windows-7/\#x1-110003.1}.
I also downloaded GSView from here:
	\web{http://pages.cs.wisc.edu/~ghost/gsview/get50.htm}.

The answer to this question looks amazing:
	\web{http://tex.stackexchange.com/questions/50827/a-simpletons-guide-to-tex-workflow-with-emacs}.

I downloaded and installed ASpell from here:
	\web{http://aspell.net/win32/}.

There are many changes that are made in the emacs init file to facilitate working with \LaTeX.

\section{Java}

Download the latest version of Java from the Oracle downloads page:
 	\web{http://www.oracle.com/technetwork/java/javase/downloads/index.html}.
To check to see if the installation was successful, open a command line and execute java -version.
This command should work.
Next execute javac -version.
If this command does not work, then the location of javac.exe needs to be added to the path.
Right-click on Computer, go to the Advanced System Settings, and click on Environment Variables.
There should be a variable named Path that we need to edit.
Select Path and click edit, or create a new variable named Path if one does not exist.
Append to the variable the file path leading to javac.
This path should end with the bin folder in the jdk that contains javac.
Afterwards check that the Path is correct by running the command javac -version again.
The command line may need to be restarted for the change in Environment Variable to take effect.

\section{Eclipse}

I use Eclipse when working with Java and Python.
This section covers how to set up Eclipse to work with Java and explains the plugins I use for Java.
For instructions using PyDev see \cref{subsec:PyDev}.

\subsection{Installation}
Eclipse can be downloaded here:
	\web{https://www.eclipse.org/downloads/}.
Simply unzip the folder and it should work; no installer should be necessary.

\subsection{Plugins}

\section{Python}

To install python, I need to install both the language itself and several additional modules.
I also need to set up PyDev with Eclipse.

\subsection{Installation}

For installation, I use the version that we use in CS 1110.
The install link is here:
	\web{http://www.cs.cornell.edu/Courses/cs1110/2014fa/materials/python.php}.
I download this version and the Cornell Extensions.
\footnote{
	I should at some point dig into the Cornell Extensions and figure out what modules and packages
		they contain.
}
I then install Komodo Edit from the same page.

\subsection{PyDev}
\label{subsec:PyDev}

\subsection{Other Python Modules}

I need to add several other python modules to their respective folders.

\subsubsection{PyLint}

\subsubsection{Coverage}

\section{Microsoft VIsual Studios}

I use the Express version because it is free.
Perhaps there is a DreamSpark thing I could do to get the Professional or Ultimate editions for free.
The download link is here:
	\web{http://www.microsoft.com/en-us/download/details.aspx?id=40787}.
Download both of the files.
Run the wdexpress\_full.exe file, which is the installer.

\end{document}
